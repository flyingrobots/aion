\sectionbreak
\section{Introduction}
\label{sec:intro}

Modern computation is built on mutable state and loosely specified
concurrency.  As systems become distributed, multi-core, and
AI-mediated, this leads to nondeterminism, opaque failure modes,
unreproducible behavior, and fundamentally incomplete provenance.

The goal of this work is to develop a mathematically precise alternative
model in which:

\begin{itemize}[leftmargin=*]
  \item global state is represented as a recursively nested,
    graph-shaped object;
  \item all evolution of that state is given by well-typed graph
    rewrites;
  \item under an explicit independence discipline and
    no-delete/no-clone-under-descent invariants, the operational
    semantics is deterministic (up to typed open graph isomorphism)
    and confluent at the level of ``ticks'' of computation
    (\cref{thm:tick-confluence,thm:two-plane,thm:global}); and
  \item the \emph{entire} interior evolution of a computation is stored
    in a compact \emph{provenance payload} attached to a single edge,
    providing an information-complete ``holographic'' encoding.
\end{itemize}

The technical starting point is algebraic graph transformation using the
double--pushout (DPO) approach in adhesive categories, together with the
Recursive Metagraph (RMG) object model we define in Section~\ref{sec:rmg}.
We extend this setting with:

\begin{enumerate}[leftmargin=*]
  \item a precise notion of RMG state and its category;
  \item a two-plane concurrent operational semantics with
    attachment--then--skeleton publication, together with
    confluence results: tick-level determinism and two-plane
    commutation (Theorems~\ref{thm:tick-confluence} and
    \ref{thm:two-plane}), and, under standard rewrite-theory
    hypotheses, global confluence (Theorem~\ref{thm:global});
  \item a provenance payload calculus giving \emph{computational
    holography};
  \item an MDL-based quasi-pseudometric on observers, the \emph{rulial
    distance}, and a correspondence between RMG derivations and
    multiway systems that clarifies the relationship to Wolfram's
    Ruliad.
\end{enumerate}

We deliberately keep the system-level \AION{} stack\footnote{%
Pronounced ``eye-ON'' (rhymes with \emph{aeon}), with stress on the second syllable.}
(AIONOS, Echo, Wesley, etc.) mostly offstage in this paper, mentioning
it only to motivate the mathematics.  The companion ``\COMPUTER{}''
paper will build on these results to define the full machine model and
operating system.

\medskip
\noindent
\textbf{Key contributions.}
The main results of this paper are:
\begin{enumerate}[leftmargin=5mm]
  \item \emph{Tick-level confluence} (Theorem~\ref{thm:tick-confluence}):
    parallel independent RMG rewrites commute, yielding per-tick
    deterministic semantics independent of scheduler serialization
    order;
  \item \emph{Two-plane commutation} (Theorem~\ref{thm:two-plane}):
    attachment and skeleton updates can be applied in either order up to
    isomorphism, via a fibration structure;
  \item \emph{Worldline uniqueness} (Corollary~\ref{cor:worldline-uniqueness}):
    global uniqueness of complete derivation worldlines up to typed
    open graph isomorphism, extending per-tick commutation to
    scheduler-independent whole runs;
  \item \emph{Computational holography} (Theorem~\ref{thm:holography}):
    the boundary data $(S_0,P)$ is information-complete with respect to
    the interior evolution, enabling reconstruction of the full
    derivation volume;
  \item \emph{Rulial distance} (Theorem~\ref{thm:rulial-triangle}):
    an MDL-based quasi-pseudometric on observers that quantifies the
    complexity of translating between different views of the same
    computation.
\end{enumerate}
