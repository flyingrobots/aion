\section*{Standing Assumptions}
\label{sec:assumptions}

For ease of reference, we summarise the main semantic assumptions
used in the determinism, holography, and rulial-distance results.

\medskip
\begin{center}
\begin{tabular}{ll}
\textbf{Assumption} & \textbf{Where used} \\[2pt]
\hline\\[-8pt]
Skeleton independence via footprints &
Defs.~\ref{def:footprint}, \ref{def:independence}, \ref{def:batch};
Thm.~\ref{thm:tick-confluence} \\
No-delete/no-clone under descent (ND/NC) &
Def.~\ref{def:no-delete}; Thm.~\ref{thm:two-plane}, Thm.~\ref{thm:global} \\
Termination / decreasing diagrams on the skeleton &
Thm.~\ref{thm:global} (conditional global confluence) \\
No re-derivation (single producer) &
Sec.~\ref{sec:holography}, Thm.~\ref{thm:backward} (backward provenance) \\
Budgeted translators and 1-Lipschitz distortion &
Sec.~\ref{sec:rulial}, Lem.~\ref{lem:rulial-basic},
Thm.~\ref{thm:rulial-triangle} (rulial distance) \\
\end{tabular}
\end{center}
\medskip

Unless otherwise stated, all results in the main text are to be read
relative to these assumptions.

