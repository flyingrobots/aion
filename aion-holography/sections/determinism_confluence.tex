\sectionbreak
\section{Determinism and Confluence}
\label{sec:determinism}

We sketch the concurrency discipline, define independence, and state the
main confluence theorems for tick-level execution.

\subsection{Scheduler state and footprints}

The global runtime state includes, in addition to the current RMG state,
a multiset of \emph{pending rewrites} with matches, footprints, and
phases.  Operationally this is described in the AION engine
specification. For the mathematics, we abstract it
as follows.

\begin{definition}[Footprint]
Given a rule $p$ and match $m : L \mono G$, define the \emph{delete set}
and \emph{use set} as
\[
  \Del(m) = m(L \setminus K), \qquad
  \Use(m) = m(L).
\]
A \emph{footprint} $F(m)$ records these sets, together with read/write
sets for attachments and a factor mask used to enforce additional
ordering constraints.
\end{definition}

\begin{definition}[Independence]
Two matches $m_1,m_2$ are (parallel) \emph{independent} if
\[
  \Del(m_1) \cap \Use(m_2) = \emptyset, \qquad
  \Del(m_2) \cap \Use(m_1) = \emptyset,
\]
and both matches satisfy the gluing conditions.  A scheduler-admissible
batch is a finite family of matches that are pairwise independent in
this sense and respect the attachment invariants.
\end{definition}

\subsection{Tick semantics and scheduler confluence}

We recall the basic setting.  Work in the adhesive category
$\OGraph_T$ of typed open graphs.  A DPOI rule is a span of monos
$p = (L \xleftarrow{\ell} K \xrightarrow{r} R)$; a match is a mono
$m : L \hookrightarrow G$ satisfying the usual gluing conditions
(dangling and identification).  A DPOI step $G \Rightarrow_p H$ is
given by the standard double square (pushout complement + pushout).

\begin{definition}[RMG state and tick]
An RMG state is a triple
\[
  \mathcal{U} \;=\; (G; \alpha, \beta)
\]
with skeleton $G \in \OGraph_T$ and attachment objects
$\alpha(v), \beta(e) \in \OGraph_T$ in the fibers over each
node $v \in V(G)$ and edge $e \in E(G)$.

A \emph{tick} consists of:
\begin{itemize}[leftmargin=*]
  \item a finite family $A$ of attachment DPOI steps in the fibers
    $\alpha(v), \beta(e)$;
  \item a finite family $S = \{(p_i,m_i)\}_{i\in I}$ of skeleton DPOI
    steps on $G$ such that the matches $\{m_i\}$ are pairwise parallel
    independent.
\end{itemize}
The tick obeys the \emph{no-delete-under-descent} invariant: if some
attachment step in $A$ touches $\alpha(v)$ or $\beta(e)$, then no
concurrent skeleton step in $S$ may delete $v$ or $e$.
Operationally, a tick publishes attachments first and then skeleton.
\end{definition}

The scheduler computes a maximal independent set of matches in the
sense above, using a safe over-approximation of $\Use\cup\Del$; we do
not repeat the implementation details here.

\begin{theorem}[Tick-level confluence]\label{thm:tick-confluence}
Let $\mathcal{U}$ be an RMG state and let
$A, S=\{(p_i,m_i)\}_{i\in I}$ form a tick as in the definition above.
Let $\sigma$ range over permutations of $I$.  For each $\sigma$, form
the serial composite
\[
   \mathcal{U}
      \;\Rewrite^{A}\;
   \mathcal{U}^{(0)}_\sigma
      \;\Rewrite^{(p_{\sigma(1)},m_{\sigma(1)})}\;
   \mathcal{U}^{(1)}_\sigma
      \;\Rewrite^{(p_{\sigma(2)},m_{\sigma(2)})}\;
   \cdots
      \;\Rewrite^{(p_{\sigma(|I|)},m_{\sigma(|I|)})}\;
   \mathcal{U}^{(|I|)}_\sigma.
\]
Then all results $\mathcal{U}^{(|I|)}_\sigma$ are isomorphic as RMG
states.  In particular, the effect of the tick is deterministic up to
typed open-graph isomorphism, independent of the serialisation order
chosen by the scheduler.
\end{theorem}

\begin{proof}
We separate the argument into the skeleton part and the attachment
part.

\emph{Skeleton.}
Consider only the skeleton matches $\{m_i : L_i \hookrightarrow G\}$.
By hypothesis, these matches are pairwise parallel independent.  By
the Concurrency / Parallel Independence Theorem for DPO rewriting in
adhesive categories (see, e.g.,~\cite{EEPT06}), parallel independent
steps commute: for any $i \neq j$ we have a diagram
\[
  G \;\Rightarrow_{(p_i,m_i)}\; G_i \;\Rightarrow_{(p_j,m'_j)}\; G_{ij}
  \quad\text{and}\quad
  G \;\Rightarrow_{(p_j,m_j)}\; G_j \;\Rightarrow_{(p_i,m'_i)}\; G_{ji}
\]
where both two-step derivations exist and the results $G_{ij}$ and
$G_{ji}$ are isomorphic.  The rewritten matches $m'_i, m'_j$ are
obtained by the standard reindexing construction of the concurrency
theorem.

We now induct on $|I|$ to obtain order-independence for the entire
family.

For $|I| = 0$ or $1$ the claim is trivial.  For $|I| = 2$ it is exactly
the commuting-square case of the concurrency theorem.

Assume the property holds for all scheduler-admissible families of
size $k$.  Let $S$ have size $k+1$.  Pick any index $j \in I$ and
factor an arbitrary serial order as
\[
  G \Rewrite^{(p_{i_1},m_{i_1})} \cdots
    \Rewrite^{(p_{i_k},m_{i_k})} G'
    \Rewrite^{(p_j,m'_j)} G''.
\]
By the induction hypothesis, the prefix of length $k$ yields a result
unique up to isomorphism, regardless of the order of the $k$ steps.
Now compare any two permutations of the full $(k+1)$ steps.  One can
be obtained from the other by a finite sequence of adjacent swaps.
Each adjacent swap exchanges two parallel independent steps; by the
two-step concurrency theorem, the corresponding length-$(k+1)$
derivations commute up to isomorphism.  Thus, by finite induction on
the number of swaps, all serialisations of the skeleton batch produce
isomorphic skeletons.

\emph{Attachments.}
Attachment steps live in the product of fibers
$\prod_{x\in V(G)\cup E(G)} \OGraph_T$, one fiber per node or edge.
Within a fixed tick, attachment steps are DPOI steps in these fibers
that do not change the skeleton $G$ itself.  Because the fibers form a
product category, DPOI steps in distinct fibers are trivially parallel
independent and commute strictly.

By the no-delete-under-descent invariant, any position $x$ whose
attachment $\alpha(x)$ or $\beta(x)$ is touched in this tick must lie
in the preserved interface $K_i$ of every concurrent skeleton step
$(p_i,m_i)$; in particular, skeleton rewriting does not delete $x$.
Skeleton steps are DPO pushouts along monos and therefore induce
reindexing isomorphisms on the fibers over preserved positions.
Hence, attachment updates may be taken to occur in a fixed copy of
each fiber and then transported along these isomorphisms without
affecting the result.

\emph{Putting the planes together.}
By definition of a tick, we always apply attachments before skeleton.
From the previous paragraph, the attachments part is independent of
the serialisation order of the skeleton part; and from the skeleton
argument, the skeleton result is independent (up to iso) of the
serialisation order of $S$.  Therefore the composite effect of the
tick is unique up to RMG isomorphism, independent of the order in
which the scheduler executes the individual steps.
\end{proof}

\subsection{Two-plane commutation via a fibration}

We now justify the two-plane discipline more structurally, using a
simple fibration view.

Let $\RMGState$ be the category of RMG states and RMG morphisms
(skeleton morphisms together with compatible fiber morphisms).  There
is a forgetful functor
\[
  \pi : \RMGState \;\longrightarrow\; \OGraph_T
\]
sending $(G;\alpha,\beta)$ to its skeleton $G$ and acting on morphisms
componentwise.  This functor is a (Grothendieck) fibration whose
fibers are products of copies of $\OGraph_T$:
\[
  \pi^{-1}(G) \;\cong\; \prod_{x\in V(G)\cup E(G)} \OGraph_T.
\]
In particular, given a mono $u : G \hookrightarrow G'$ in the base,
there is a reindexing functor
\[
  u^\ast : \pi^{-1}(G') \longrightarrow \pi^{-1}(G)
\]
which transports attachments along $u$ by precomposition.

An \emph{attachment step} is a DPOI step in some fiber
$\pi^{-1}(G)$; a \emph{skeleton step} is a DPOI step in the base
$\OGraph_T$.  Both are built from pushouts along monos.

\begin{theorem}[Two-plane commutation]\label{thm:two-plane}
Let $\mathcal{U} = (G;\alpha,\beta)$ be an RMG state.  Let
$A : \mathcal{U} \Rewrite \mathcal{U}_A$ be a finite composite of
attachment steps in the fiber over $G$, and let
$S : G \Rewrite G'$ be a composite of skeleton DPOI steps such that
no step in $S$ deletes or clones any position whose attachment is
touched by $A$ (no-delete/no-clone under descent).  Then there exists
an attachment composite $A' : (G';\alpha',\beta') \Rightarrow
(G';\alpha'',\beta'')$ in the fiber over $G'$ such that the following
square in $\RMGState$ commutes up to isomorphism:
\[
\begin{tikzcd}
  (G;\alpha,\beta) \arrow[r,"A"] \arrow[d,"S"']
    & (G;\alpha_A,\beta_A) \arrow[d,"S'"] \\
  (G';\alpha',\beta') \arrow[r,"A'"']
    & (G';\alpha'',\beta'')
\end{tikzcd}
\]
In particular, applying attachments then skeleton yields the same
result (up to iso) as applying skeleton then the transported
attachments.
\end{theorem}

\begin{figure}[t]
  \centering
  \begin{tikzpicture}[
      node distance=32mm,
      obj/.style={rectangle,draw=teal!70!black,fill=teal!8,thick,minimum width=22mm,
                  minimum height=8mm,align=center},
      arr/.style={-Latex,thick,teal!70!black},
      >=Latex
    ]

    \node[obj] (U)   {$(G;\alpha,\beta)$};
    \node[obj,right=of U] (UA)  {$(G;\alpha_A,\beta_A)$};
    \node[obj,below=of U] (US)  {$(G';\alpha',\beta')$};
    \node[obj,below=of UA] (USA) {$(G';\alpha'',\beta'')$};

    \draw[arr] (U)  -- node[above]{\scriptsize attachments $A$} (UA);
    \draw[arr] (U)  -- node[left]{\scriptsize skeleton $S$} (US);
    \draw[arr] (UA) -- node[right]{\scriptsize skeleton $S'$} (USA);
    \draw[arr] (US) -- node[below]{\scriptsize transported $A'$} (USA);

  \end{tikzpicture}
  \caption{Two-plane commutation: attachment updates $A$ in the fiber over $G$ commute with skeleton rewriting $S$ in the base, up to transporting the attachment steps along the skeleton morphism. Theorem~\ref{thm:two-plane} shows that the two paths in the square yield isomorphic RMG states.}
  \label{fig:two-plane-square}
\end{figure}

\begin{proof}
Write the skeleton composite $S$ as a sequence of DPOI steps
\[
  G = G_0 \Rewrite G_1 \Rewrite \cdots \Rewrite G_n = G'.
\]
Each step $G_{k-1} \Rewrite G_k$ is a pushout along a mono in
$\OGraph_T$.  Because $\OGraph_T$ is adhesive, pushouts along monos
are Van Kampen squares and stable under pullback~\cite{LS06}.

\smallskip
\noindent
The no-delete/no-clone-under-descent hypothesis ensures that every
position $x$ whose attachment is touched by $A$ lies in the preserved
interface of each skeleton step.  Thus, along the composite mono
$u : G \hookrightarrow G'$, the reindexing functor
$u^\ast : \pi^{-1}(G') \to \pi^{-1}(G)$ is an isomorphism on the
fibers corresponding to positions touched by $A$; informally, the
skeleton only renames those attachment slots.

\smallskip
\noindent
Consider first a single attachment step in the fiber over some
position $x$:
\[
  (G;\alpha,\beta) \Rightarrow_A (G;\alpha_A,\beta_A),
\]
given by a DPOI double square in the corresponding component of the
fiber $\pi^{-1}(G) \cong \prod_x \OGraph_T$.  Forming the pullback of
this square along the mono $u : G \hookrightarrow G'$ yields a square
in the fiber over $G'$; by stability of pushout complements and
Van Kampen, this square is again a DPOI step, which we denote by
$A'$:
\[
  (G';\alpha',\beta') \Rightarrow_{A'} (G';\alpha'',\beta'').
\]
At the level of the total category $\RMGState$ we thus obtain a
commuting cube whose back face is the original attachment step, whose
bottom face is the skeleton step, and whose front face is the
transported attachment step.  All vertical faces are pullbacks and all
horizontal faces are pushouts along monos; Van Kampen ensures that the
top and bottom composites are isomorphic.

\smallskip
\noindent
Iterating this construction over the finite families of attachment and
skeleton steps yields a composite cube whose front and back faces are
the two composites $S\circ A$ and $A'\circ S'$ in the statement.  By
pasting of Van Kampen squares, the induced morphism between the top
and bottom objects is an isomorphism in $\RMGState$.  Hence the
diagram commutes up to isomorphism.
\end{proof}

\subsection{Global confluence}

To obtain global Church--Rosser properties for the entire rewrite
system, additional hypotheses are required.  We invoke the standard
critical-pair lemma and Newman's lemma for terminating systems:

\begin{theorem}[Conditional global confluence]\label{thm:global}
Let $R$ be a finite DPOI rule set.  Suppose that:
\begin{enumerate}[leftmargin=*]
  \item every DPOI critical pair of $R$ is joinable (modulo boundary
    isomorphism), and
  \item the induced rewrite relation is terminating on the class of
    states considered, or admits a decreasing-diagrams labelling.
\end{enumerate}
Then the rewrite relation is confluent.
\end{theorem}

This theorem applies directly to the skeleton plane; together with the
attachment invariants and two-plane commutation, it yields uniqueness of
\emph{worldlines} at the level of RMG states when the rule pack
satisfies these conditions.

The upshot is that, under explicit and checkable rule-pack assumptions,
the runtime has a unique deterministic evolution from any given initial
state.  This is the key precondition for holographic provenance: there
is exactly one interior history to encode.