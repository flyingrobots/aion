\sectionbreak
\section{Rulial Distance: A Computable Metric on Observer Space}
\label{sec:rulial}

We next formalize observers and an MDL-based distance between them, the
\emph{rulial distance}.  This provides a geometry on different
descriptions of the same underlying RMG universe.

\subsection{Observers as functors}

Fix an RMG universe $(U,R)$ and its history category
$\Hist(U,R)$ whose objects are states and whose morphisms are derivation
paths between them.  An \emph{observer} is a functor
\[
  O : \Hist(U,R) \To \mathcal{Y},
\]
where $\mathcal{Y}$ is a suitable category of observations (symbol
streams, trace graphs, etc.), subject to resource budgets on time and
memory.

Different observers may:

\begin{itemize}[leftmargin=*]
  \item choose different projections of the same wormhole payloads;
  \item aggregate or forget structure;
  \item expose different notions of causality.
\end{itemize}

\subsection{Translators and MDL cost}

A \emph{translator} between observers $O_1$ and $O_2$ is a functorial
construction $T : O_1 \Rewrite O_2$ realized as a small DPOI
transducer, together with a distortion metric on outputs.

\paragraph{Example (SQL$\leftrightarrow$AST translator).}
Consider two observers of an RMG universe modeling a database
query planner.  Observer $O_1$ sees the wormhole payload $P$ as
a sequence of AST transformations (parse tree $\to$ optimized AST
$\to$ query plan), while observer $O_2$ sees only the initial SQL
string and final execution trace.  A translator $T_{12}$ must
reconstruct the SQL from the AST evolution: it can parse the initial
AST root, emit the corresponding SQL, and summarize the execution
steps by their side effects.  The reverse translator $T_{21}$ parses
the SQL and heuristically infers an AST evolution consistent with the
execution trace, incurring some distortion.  The description lengths
$\mathrm{DL}(T_{12}), \mathrm{DL}(T_{21})$ and distortion costs
quantify how ``close'' these two viewpoints are in rulial space.

Let $\mathrm{DL}(T)$ be a prefix-code description length for $T$
(MDL cost), and let $\mathrm{Dist}$ be a distortion metric on observed
traces.  Intuitively, we measure how far two observers are by asking:
how complex is a program that translates the traces of $O_1$ into those
of $O_2$ (and back again), and how much distortion does that translation
incur?  Minimum Description Length (MDL) gives a principled way to
quantify program complexity as a description length in bits; we combine
this with a distortion metric on traces to obtain our notion of rulial
distance.

We then define a symmetric cost:
\[
  D_{\tau,m}(O_1,O_2)
   = \inf_{T_{12},T_{21}}
      \Bigl(
        \mathrm{DL}(T_{12}) + \mathrm{DL}(T_{21})
        + \lambda \bigl(
           \mathrm{Dist}(O_2, T_{12}\circ O_1) +
           \mathrm{Dist}(O_1, T_{21}\circ O_2)
        \bigr)
      \Bigr)
\]
under time and memory budgets $(\tau,m)$.

\begin{proposition}[Pseudometric]
$D_{\tau,m}$ is a pseudometric on observers: it is nonnegative,
symmetric, and $D_{\tau,m}(O,O)=0$ for all $O$.
\end{proposition}

\begin{theorem}[Triangle inequality for rulial distance]\label{thm:rulial-triangle}
Assume:
\begin{enumerate}[leftmargin=*]
  \item the description length $\mathrm{DL}$ is based on a prefix code
    and satisfies, for some constant $c\ge 0$,
    \[
      \mathrm{DL}(T_{13})
      \le \mathrm{DL}(T_{12}) + \mathrm{DL}(T_{23}) + c
    \]
    whenever $T_{13}$ is a composition of translators $T_{23}\circ T_{12}$;
  \item the distortion measure $\mathrm{Dist}$ is a metric on observer
    traces, hence satisfies the usual triangle inequality.
\end{enumerate}
Then the MDL-based rulial distance $D_{\tau,m}$ is a pseudometric and
satisfies a triangle inequality up to an additive constant:
\[
  D_{\tau,m}(O_1,O_3)
  \le D_{\tau,m}(O_1,O_2) + D_{\tau,m}(O_2,O_3) + 2c.
\]
\end{theorem}

\begin{proof}
Nonnegativity and symmetry are immediate from the definition of
$D_{\tau,m}$ as an infimum over sums of nonnegative symmetric terms.

For the triangle inequality, fix $\varepsilon>0$ and choose near-optimal
translators $(T_{12},T_{21})$ and $(T_{23},T_{32})$ attaining the
infima for $D_{\tau,m}(O_1,O_2)$ and $D_{\tau,m}(O_2,O_3)$ up to
$\varepsilon/2$.  Form composite translators
$T_{13}=T_{23}\circ T_{12}$ and $T_{31}=T_{21}\circ T_{32}$.
By the subadditivity of $\mathrm{DL}$,
\[
  \mathrm{DL}(T_{13}) \le \mathrm{DL}(T_{12})+\mathrm{DL}(T_{23})+c,
  \qquad
  \mathrm{DL}(T_{31}) \le \mathrm{DL}(T_{21})+\mathrm{DL}(T_{32})+c.
\]
By the triangle inequality for $\mathrm{Dist}$,
\[
  \mathrm{Dist}(O_3,T_{13}\circ O_1)
  \le \mathrm{Dist}(O_3,T_{23}\circ O_2)
     +\mathrm{Dist}(O_2,T_{12}\circ O_1),
\]
and similarly with roles reversed.

Summing these bounds and using the near-optimality of the chosen
translators yields
\[
  D_{\tau,m}(O_1,O_3)
  \le D_{\tau,m}(O_1,O_2) + D_{\tau,m}(O_2,O_3) + 2c + \varepsilon.
\]
Since $\varepsilon>0$ was arbitrary, the inequality without $\varepsilon$
follows.
\end{proof}

The quantity $D_{\tau,m}$ is the \emph{rulial distance} between
observers: it measures how hard it is to translate between descriptions
of the same underlying history.  Observers with small distance live in
nearby ``frames''; those with large distance inhabit distant regions of
the Ruliad.

\subsection{Observer projections of wormholes}

Given a wormhole $(S_0,P)$, different observers may:

\begin{itemize}[leftmargin=*]
  \item expose only coarse-grained stages of $P$ (e.g.\ AST$\to$IR$\to$SQL);
  \item restrict to semantic effects (e.g.\ DB schema, invariants);
  \item highlight only adversarial branches;
  \item or inspect every microstep.
\end{itemize}

The holographic encoding thus supports a wide range of observer
perspectives from a single payload.

\begin{figure}[t]
  \centering
  \begin{tikzpicture}[
      wormhole/.style={rectangle,draw=green!60!black,fill=green!5,thick,rounded corners,
                       minimum width=32mm,minimum height=14mm,align=center},
      observer/.style={rectangle,draw=blue!70!black,fill=blue!8,thick,rounded corners=3pt,
                       minimum width=18mm,minimum height=8mm,align=center,font=\small},
      arrow/.style={-Latex,thick},
      >=Latex
    ]

    % Central wormhole
    \node[wormhole] (W) at (0,0)
      {wormhole\\[-1pt]
       \scriptsize $(S_0,P)$};

    % Observers
    \node[observer] (O1) at (-3.5,2.2) {$O_1$\\[-2pt]\scriptsize coarse stages};
    \node[observer] (O2) at (3.5,2.2) {$O_2$\\[-2pt]\scriptsize semantic};
    \node[observer] (O3) at (-3.5,-2.2) {$O_3$\\[-2pt]\scriptsize adversarial};
    \node[observer] (O4) at (3.5,-2.2) {$O_4$\\[-2pt]\scriptsize full microsteps};

    % Projections
    \draw[arrow,blue!70!black] (W.north west) -- (O1.south east);
    \draw[arrow,blue!70!black] (W.north east) -- (O2.south west);
    \draw[arrow,blue!70!black] (W.south west) -- (O3.north east);
    \draw[arrow,blue!70!black] (W.south east) -- (O4.north west);

    % Labels on arrows
    \node[rotate=45,font=\scriptsize] at (-1.8,1.2) {project};
    \node[rotate=-45,font=\scriptsize] at (1.8,1.2) {project};
    \node[rotate=-45,font=\scriptsize] at (-1.8,-1.2) {project};
    \node[rotate=45,font=\scriptsize] at (1.8,-1.2) {project};

  \end{tikzpicture}
  \caption{Multiple observers projecting the same wormhole $(S_0,P)$
  into different trace formats.  Each observer $O_i$ extracts a
  different view of the interior evolution: coarse-grained stages,
  semantic invariants, adversarial branches, or full microsteps.
  The rulial distance measures the complexity of translating
  between these views.}
  \label{fig:observer-projections}
\end{figure}