\section*{Standing Assumptions}
\label{sec:assumptions}

For ease of reference, we summarise the main semantic assumptions
used in the determinism, holography, and rulial-distance results.

\medskip
\begin{center}
\begin{tabular}{>{\raggedright\arraybackslash}p{0.37\textwidth}>{\raggedright\arraybackslash}p{0.55\textwidth}}
\toprule
\textbf{Assumption} & \textbf{Where used} \\
\midrule
Skeleton independence via footprints &
Definitions~\ref{def:footprint}--\ref{def:batch};
Theorem~\ref{thm:tick-confluence} \\
No-delete/no-clone-under-descent (ND/NC) &
Definition~\ref{def:no-delete}; Theorems~\ref{thm:two-plane} and~\ref{thm:global} \\
Termination / decreasing diagrams on the skeleton &
\cref{thm:global} (conditional global confluence) \\
No re-derivation (single producer) &
\cref{sec:holography}, \cref{thm:backward} (backward provenance) \\
Budgeted translators and 1-Lipschitz distortion &
\cref{sec:rulial}, Lemma~\ref{lem:rulial-basic} and Theorem~\ref{thm:rulial-triangle} (rulial distance) \\
\bottomrule
\end{tabular}
\end{center}
\medskip

Unless otherwise stated, all results in the main text are to be read
relative to these assumptions.
